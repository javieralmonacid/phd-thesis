%   DOCUMENT CLASS  %%%%%%%%%%%%%%%%%%%%%%%%%%%%%%%%%%%%%%%%%%%%%%%%%%%%%%%%%%%
%
%   Use the `sfuthesis` class to format your thesis.
%
%   For more information about thesis formatting requirements, go to
%   http://www.lib.sfu.ca/help/publish/thesis or ask a thesis advisor at the
%   SFU Research Commons.


\documentclass{sfuthesis}



%   DOCUMENT METADATA  %%%%%%%%%%%%%%%%%%%%%%%%%%%%%%%%%%%%%%%%%%%%%%%%%%%%%%%%
%
%   Fill in the following information for the title page and declaration of
%   committee page. Please review the Declaration of Committee page
%   instructions on the library's thesis website before completing this page:
%   https://www.lib.sfu.ca/help/publish/thesis/format/declaration-committee

%   Choose the \faculty entry below from the following list:
%
%       - Faculty of Applied Sciences
%       - Faculty of Arts and Social Sciences
%       - Beedie School of Business
%       - Faculty of Communication, Art and Technology
%       - Faculty of Education
%       - Faculty of Environment
%       - Faculty of Health Sciences
%       - Faculty of Science

\title{Computational strategies for the nonlinear elastodynamics of skeletal muscle tissue}
\thesistype{Thesis}
\author{Javier Alejandro Almonacid Paredes}
\previousdegrees{%
    M.Sc., Simon Fraser University, 2020\\
    B.Sc., Universidad de Concepci\'{o}n, 2015}
\degree{Doctor of Philosophy}
\department{Department of Mathematics}
\faculty{Faculty of Science}
\copyrightyear{2025}
\semester{Spring 2025}


%   You may include up to six keywords or phrases. Keywords should be separated
%   with semicolons. No punctuation at the end.
\keywords{thesis template; Simon Fraser University; \LaTeX; time travel paradoxes}

\committee{
    \chair{TBD}{Professor \\ Department of Mathematics}
    \member{Nilima Nigam}{Supervisor \\ Professor \\ Department of Mathematics}
    \member{James Wakeling}{Committee Member (?)\\ Professor \\ Department of Biomedical Physiology and Kinesiology}
    \member{TBD}{Internal Examiner \\ Professor \\ Department of Mathematics}
    \member{TBD}{External Examiner \\ Professor \\ Department of Quantum Fields \\ Mars University}
}



%   PACKAGES %%%%%%%%%%%%%%%%%%%%%%%%%%%%%%%%%%%%%%%%%%%%%%%%%%%%%%%%%%%%%%%%%%
%
%   Add any packages you need for your thesis here.
%   You don't need to call the following packages, which are already called in
%   the sfuthesis class file:
%
%   - appendix
%   - etoolbox
%   - fontenc
%   - geometry
%   - lmodern
%   - nowidow
%   - setspace
%   - tocloft
%
%   If you call one of the above packages (or one of their dependencies) with
%   options, you may get a "Option clash" LaTeX error. If you get this error,
%   you can fix it by removing your copy of \usepackage and passing the options
%   you need by adding
%
%       \PassOptionsToPackage{<options>}{<package>}
%
%   before \documentclass{sfuthesis}.
%
%   The following packages are a few suggestions you might find useful.
%
%   (1) amsmath and amssymb are essential if you have math in your thesis;
%       they provide useful commands like ``blackboard bold'' symbols and
%       environments for aligning equations.
%   (2) amsthm includes allows you to easily change the style and numbering of
%       theorems. It also provides an environment for proofs.
%   (3) graphicx allows you to add images with \includegraphics{filename}.
%   (4) hyperref turns your citations and cross-references into clickable
%       links, and adds metadata to the compiled PDF.
%   (5) pdfpages lets you import pages of external PDFs using the command
%       \includepdf{filename}. You will need to do this if your research
%       requires an Ethics Statement.
%

\usepackage{amsmath}                            % (1)
\usepackage{amssymb}                            % (1)
\usepackage{amsthm}                             % (2)
\usepackage{graphicx}                           % (3)
\usepackage[pdfborder={0 0 0}]{hyperref}        % (4)
% \usepackage{pdfpages}                         % (5)
% ...
% ...
% ...
% ... add your own packages here!

\usepackage{background}
\backgroundsetup{
  position=current page.north,
  angle=0,
  nodeanchor=north,
  vshift=-10mm,
  opacity=1,
  scale=1.5,
  contents=Submitted for revision on \today
}


%   OTHER CUSTOMIZATIONS %%%%%%%%%%%%%%%%%%%%%%%%%%%%%%%%%%%%%%%%%%%%%%%%%%%%%%
%
%   Add any packages you need for your thesis here. We've started you off with
%   a few suggestions.
%
%   (1) Use a single word space between sentences. If you disable this, you
%       will have to manually control spacing around abbreviations.
%   (2) Correct the capitalization of "Chapter" and "Section" if you use the
%       \autoref macro from the `hyperref` package.
%   (3) The LaTeX thesis template defaults to one-and-a-half line spacing. If
%       your supervisor prefers double-spacing, you can redefine the
%       \defaultspacing command.
%

\frenchspacing                                    % (1)
\renewcommand*{\chapterautorefname}{Chapter}      % (2)
\renewcommand*{\sectionautorefname}{Section}      % (2)
\renewcommand*{\subsectionautorefname}{Section}   % (2)
% \renewcommand{\defaultspacing}{\doublespacing}  % (3)
% ...
% ...
% ...
% ... add your own customizations here!

\usepackage{bm,bbm}

%-----------------------------------------------
% Eliminate ugly boxes around references.
\usepackage{xcolor}
\hypersetup{
    colorlinks,
    linkcolor={red!50!black},
    citecolor={blue!50!black},
    urlcolor={blue!80!black}
}
%------------------------------------------------

\numberwithin{equation}{chapter}
\numberwithin{figure}{chapter}
\numberwithin{table}{chapter}


\newtheorem{objective}{Objective}
\renewcommand*{\theobjective}{\Alph{objective}}
\newtheorem{theorem}{Theorem}[chapter]
\newtheorem{remark}[theorem]{Remark}

\theoremstyle{definition}
\newtheorem{definition}{Definition}[chapter]
\newtheorem{lemma}[definition]{Lemma}
\newtheorem{example}[definition]{Example}

%%%%%%%%%%%%%%%%%%%%%%%%%%%%%%%%%%%%%%%%%%%%%%%%%%%
%
%  DEFINITIONS 
%
%%%%%%%%%%%%%%%%%%%%%%%%%%%%%%%%%%%%%%%%%%%%%%%%%%%

\def\*#1{{\mathbf{#1}}} % bold letters!
\newcommand{\pder}[2]{\dfrac{\partial #1}{\partial #2}}
\newcommand{\Dder}[2]{\dfrac{\mathrm{D} #1}{\mathrm{D} #2}}
\newcommand{\der}[2]{\dfrac{\mathrm{d} #1}{\mathrm{d} #2}}
\newcommand{\dder}[2]{\dfrac{\mathrm{d} #1}{\mathrm{d} #2}}
\newcommand{\divs}[1]{{\mathrm{div} \, #1}}
\newcommand{\Divs}[1]{{\mathrm{Div} \, #1}}
\newcommand{\divt}[1]{{\bm{\mathrm{div}} \, #1}}
\newcommand{\Divt}[1]{{\bm{\mathrm{Div}} \, #1}}

\newcommand{\R}{\mathbb{R}}
\newcommand{\B}{\mathcal{B}}
\newcommand{\F}{\mathcal{F}}
\newcommand{\I}{{\bar{I}}}
\newcommand{\FF}{{\bm{\mathcal{F}}}}
\newcommand{\C}{\mathbb{C}}
\newcommand{\T}{\top}
\renewcommand{\c}{\mathbbm{c}}
\renewcommand{\P}{\mathbb{P}}
\newcommand{\p}{\mathbbm{p}}
\newcommand{\vphi}{\varphi}

\def\bsigma{{\bm{\sigma}}}
\def\btau{{\bm{\tau}}}
\def\bchi{{\bm{\chi}}}
\def\bxi{{\bm{\xi}}}
\def\bphi{{\bm{\varphi}}}

\newcommand{\javicomment}[1]{\noindent {\color{red}\textbf{Comment by Javi: #1}}}

%%%%%%%%%%%%%%%%%%%%%%%%%%%%%%%%%%%%%%%%%%%%%%%%%%%
%
% END OF DEFINITIONS
%
%%%%%%%%%%%%%%%%%%%%%%%%%%%%%%%%%%%%%%%%%%%%%%%%%%%



%   FRONTMATTER  %%%%%%%%%%%%%%%%%%%%%%%%%%%%%%%%%%%%%%%%%%%%%%%%%%%%%%%%%%%%%%
%
%   Title page, committee page, abstract, dedication, acknowledgements, table
%   of contents, etc.
%
%   If your research requires an Ethics Statement, download the pdf from
%   https://www.lib.sfu.ca/help/publish/thesis/regulations#ethics-statement
%   to your thesis folder, then uncomment the appropriate lines below.

\begin{document}

\frontmatter
\maketitle{}
\makecommittee{}

%\addtoToC{Ethics Statement}%
%\includepdf[pagecommand={\thispagestyle{plain}}]{ethics_statement_piii.pdf}%
%\clearpage

\begin{abstract}
Abstract paragraphs should be unindented. Master's abstracts are limited to 150 words; the limit is 350 words for doctoral abstracts. Abstract text must fit on a single page.
\end{abstract}


\begin{dedication}
This is an optional page. Use your choice of paragraph style for text on this page.
\end{dedication}


\begin{acknowledgements}
This is an optional page. Use your choice of paragraph style for text on this page.
\end{acknowledgements}

\addtoToC{Table of Contents}%
\hypersetup{linkbordercolor=black,hidelinks}
\tableofcontents%
\clearpage

%   This is an optional page. Remove the following lines if you don't have any tables.
\addtoToC{List of Tables}%
\listoftables%
\clearpage

%   This is an optional page. Remove the following lines if you don't have any figures.
\addtoToC{List of Figures}%
\listoffigures%
\clearpage





%   MAIN MATTER  %%%%%%%%%%%%%%%%%%%%%%%%%%%%%%%%%%%%%%%%%%%%%%%%%%%%%%%%%%%%%%
%
%   Start writing your thesis --- or start \include ing chapters --- here.
%

\mainmatter%

\chapter{Introduction}

Start writing or pasting in your text here. By default, only works cited in the text will be added to the bibliography~\cite{HolzapfelBook}.

\section{Physiology of skeletal muscle}

\section{State of the art in muscle modelling}

\section{Continuum mechanics of deformable solids}

This might be a long section, maybe could be a chapter at the beginning of part 2? Should include some comments on generic deformation $\rho_0 \*u_{tt} = \Divt{\*P} + \*f$ and reduction to 1D if this is left as a section in the Introduction.

\chapter{Efficient solvers for multi-body models of the muscle-tendon unit}

Work with Evan on mass enhanced muscle models. We need to discuss what would be appropriate to show here, besides the model and the numerical strategy (dynamic/mass effects? Focus on one cycle? Effects of different muscles and activities?).
Github repository: mass-enhanced-muscle-models.

\chapter{Physiological stability of multi-body models}

Show that the mass enhanced muscle model yields a 1D PDE. Augment the PDE with a Neo-Hookean term. Bring back the PDE to a system of ODEs and show that this model is physiologically stable.
Github repository: derived from mass-enhanced-muscle-model -> separate repository?


\chapter{First order is enough: time discretization alternatives for dynamic Neo-Hookean deformation}

Newmark methods, Rothe's vs. method of lines comparisons.
Github repository: one-dimensional-muscle-models

\part{Three-dimensional elastodynamics}

\chapter{A Lagrangian framework for a model of skeletal muscle tissue}

Sections to include:
\begin{enumerate}
    \item Lagrangian model
    \item Fully-implicit time discretization
    \item Linearization strategies
\end{enumerate}

\chapter{Flexodeal: a new finite-element tool for studying musculoskeletal dynamics}

\begin{enumerate}
    \item Implementation details
    \item Convergence
    \item Nonlinear Solvers
    \item Different muscle architectures (e.g. bi-pennate)
\end{enumerate}

\chapter{Gearing in muscle tissues: a first-of-its-kind computational study}

From paper w/ James. Add different pennations. Cedar access?



\chapter{Conclusions and future work}


%   BACK MATTER  %%%%%%%%%%%%%%%%%%%%%%%%%%%%%%%%%%%%%%%%%%%%%%%%%%%%%%%%%%%%%%
%
%   References and appendices. Appendices come after the bibliography and
%   should be in the order that they are referred to in the text.
%
%   If you include figures, etc. in an appendix, be sure to use
%
%       \caption[]{...}
%
%   to make sure they are not listed in the List of Figures.
%

\backmatter%
    \addtoToC{Bibliography}
    \bibliographystyle{siam}
    \bibliography{references}

\begin{appendices} % optional

\chapter{Force relationships}

Describe in detail force-length and force-velocity relationships in use.
Github repository: force-relationships

\chapter{Strain energies}

Include all strain energies: base material, ecm, cellular material, fat, aponeurosis/tendon.

\end{appendices}
\end{document}
